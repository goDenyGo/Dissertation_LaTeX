\chapter{ПЕРЕЛІК СКОРОЧЕНЬ, УМОВНИХ ПОЗНАК}
% За алфавітом

\begin{description}
\item[AMD64/EM64T] --- 64-розрядна мікропроцесорна архітектура
\item[Cloud computing] --- технологія обробки даних на базі Internet-сервісів
\item[CPU] --- центральний процесор
\item[Gbps] --- гігабіт за секунду
\item[GHz] --- гігагерц
\item[GPU] --- графічний процесор
\item[IBM] --- американська корпорація, виробник обчислювальної техніки
\item[Infiniband, Myrinet, Quadrix] --- високопродуктивні мережі передачі даних
\item[Linpack] --- бібліотека програм для тестування комп'ютерів
\item[Linux] --- UNIX- подібна операційна система
\item[Lustre] --- розподілена файлова система
\item[RAID] --- надлишковий масив незалежних дисків
\item[RAM] --- запам'ятовуючий пристрій з довільним доступом
\item[SLURM] --- менеджер ресурсів кластера
\item[Web-портал] --- Web-сайт, що надає користувачам різноманітні інтерактивні сервіси
\item[БПЗ] --- базове програмне забезпечення
\item[ЕОМ] --- електронна обчислювальна машина
\item[ІК НАНУ] --- Інститут кібернетики ім. В.М.~Глушкова НАН України
\item[ОС] --- операційна система
\item[ПЗ] --- програмне забезпечення
\item[СКІТ] --- суперкомп'ютер інформаційних технологій
\item[СЛАР] --- система лінійних алгебраїчних рівнянь
\item[МСЕ] --- метод скінченних елементів
\item[суперкомп'ютер] --- обчислювальна система, яка має визначні характеристики продуктивності у порівнянні з комп'ютерами загального вжитку, у роботі терміни ``суперкомп'ютер'', ``обчислювальний кластер'', ``обчислювальний комплекс'' використовуються як синоніми
\item[гібридний вузол] --- обчислювальний вузол, в якому для обрахунків використовуються як універсальні процесори, так і спеціалізовані акселератори
\end{description}