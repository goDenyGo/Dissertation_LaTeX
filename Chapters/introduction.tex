\chapter{ВСТУП}

\section{Актуальність теми} 

Захворювання печінки, зокрема пухлинні ураження, становлять одну з найактуальніших проблем сучасної медицини, оскільки вони суттєво впливають на якість і тривалість життя пацієнтів. Серед злоякісних новоутворень найбільш поширеними є гепатоцелюлярна карцинома (ГЦК) та холангіокарцинома, які характеризуються агресивним перебігом, обмеженими можливостями системної терапії та високим ризиком рецидиву після лікування. Вторинні злоякісні ураження печінки, зокрема метастази колоректального раку, також мають велике клінічне значення, оскільки печінка є одним з основних органів-мішеней для гематогенного поширення пухлин.
Окрім злоякісних уражень, значну частку патології печінки становлять доброякісні пухлини, такі як гемангіоми, фокальна нодулярна гіперплазія та аденоми. Хоча вони часто мають безсимптомний перебіг і в більшості випадків не потребують хірургічного втручання, у певних клінічних ситуаціях – наприклад, при великих розмірах пухлини, ризику розриву або здавленні суміжних структур – оперативне лікування є необхідним. Традиційно резекції печінки виконувалися відкритим доступом, що забезпечувало радикальність втручання, однак було пов’язано з високою травматичністю, значною крововтратою, тривалим післяопераційним періодом та високою частотою ускладнень.
У зв’язку з цим лапароскопічний підхід у лікуванні пухлин печінки привертає все більшу увагу як у контексті злоякісних, так і доброякісних уражень. За останні десятиліття методика лапароскопічної резекції печінки значно вдосконалилася, що дозволило розширити її застосування. Низка досліджень демонструє, що лапароскопічна хірургія печінки забезпечує меншу операційну травматичність, зменшення післяопераційного болю, скорочення періоду госпіталізації та швидше відновлення пацієнтів порівняно з відкритими резекціями.
Попри численні переваги, широке впровадження лапароскопічних резекцій печінки гальмується рядом факторів. Це, зокрема, висока технічна складність операцій, анатомічні особливості печінкової васкуляризації, ризик інтраопераційної кровотечі, необхідність використання спеціалізованого обладнання та високий рівень кваліфікації хірурга. Крім того, питання вибору між відкритим та лапароскопічним доступом залишається предметом дискусій, особливо у випадках великих або складно локалізованих пухлин.
На сьогодні відсутні єдині стандартизовані критерії щодо вибору оптимального підходу до лікування пухлин печінки, що створює певну невизначеність у клінічній практиці. Більшість рішень щодо застосування лапароскопічного методу базуються на досвіді окремих центрів та хірургів, що ускладнює об’єктивне оцінювання ефективності методу. Саме тому необхідним є подальше дослідження результатів лапароскопічних резекцій печінки, їх порівняння з відкритими втручаннями та розробка алгоритмів вибору хірургічної тактики для різних категорій пацієнтів.
Додатковим аспектом, що потребує вивчення, є вплив лапароскопічного підходу на післяопераційний період, частоту ускладнень, відновлення функціонального стану пацієнтів та їхню якість життя. Використання стандартизованих опитувальників та об’єктивних методів оцінки дозволить визначити переваги та можливі недоліки цього методу з урахуванням специфіки кожного клінічного випадку.
Таким чином, обґрунтування доцільності та ефективності лапароскопічного підходу до хірургічного лікування пухлин печінки є актуальним завданням сучасної хірургії. Результати цього дослідження сприятимуть вдосконаленню хірургічної тактики, розширенню показань до лапароскопічних втручань, зниженню рівня ускладнень та покращенню результатів лікування пацієнтів із пухлинною патологією печінки.


\section{Зв’язок роботи з науковими програмами, планами, темами.}

Дисертаційна робота виконана відповідно до тематичного плану науково-дослідних робіт Національного інституту хірургії та трансплантології імені О. О. Шалімова НАМН України і є фрагментом комплексної теми «НАЗВА ТЕМИ» (номер державної реєстрації ХХХХХХХХ)

\section{Мета і завдання дослідження} 


\textbf{Метою дослідження} є підвищення ефективності та безпеки лікування хворих на злоякісні та доброякісні пухлини печінки шляхом розробки патогенетично обгрунтованого диференційованого підходу до застосування лапароскопічних резекцій печінки та вдосконалення існуючої техніки їх проведення.



Відповідно до поставленої мети сформульовані наступні \textbf{завдання дослідження}:
\begin{enumerate}
    \item Дослідити патогенетичні особливості злоякісних та доброякісних пухлин печінки, що впливають на вибір хірургічної тактики та можливість застосування лапароскопічних резекцій
    %%% (1)Огляд літератури
    
    
    \item Оцінити діагностичні можливості методів обстеження хворих із злоякісними та доброякісними пухлинами печінки та розробити алгоритм передопераційного обстеження для визначення оптимального типу втручання
    %%% (2) Геометрія черевної порожнини, якість печінки 
    
    
    \item Оптимізувати покази та протипокази до лапароскопічних резекцій печінки при різних формах злоякісних та доброякісних пухлин печінки, в залежності від розмірів пухлини, її локалізації та стадії захворювання
    %%% (3) Положення пацієнта, троакарні точки доступу
    
    \item Вдосконалити техніку існуючих методів лапароскопічних резекцій з приводу злоякісних та доброякісних пухлин печінки шляхом впровадження нових технічних прийомів
    %%% (4) Обгрунтування вибору 
    
    
    \item Вивчити вплив лапароскопічних резекцій на післяопераційний період, частоту ускладнень, час госпіталізації та відновлення пацієнтів
    %%% (5) Аналіз післяопераційних ускладненнь (порівняння відкритих та лапароскопічних) 
    
    
    \item Оцінити якість життя пацієнтів після лапароскопічних резекцій печінки за допомогою стандартизованих опитувальників до та після операції
    %%% (6) Онкологічна ефективність (порівняння відкритих та лапароскопічних)
    
    
    \item Провести порівняльний аналіз результатів лапароскопічних і відкритих резекцій печінки з точки зору онкологічної ефективності
    %%% (7) Якість життя (порівняння відкритих та лапароскопічних)
    
    
    \item Розробити алгоритм вибору хірургічної тактики для пацієнтів зі злоякісними та доброякісними пухлинами печінки, враховуючи індивідуальні особливості перебігу захворювання та можливість використання лапароскопічного підходу
    %%% (8) Критерії вибору відкрита/лапароскопічна на основі 
    
    
\end{enumerate}


\textbf{Об’єкт дослідження:} злоякісні та доброякісні пухлини печінки

\textbf{Предмет дослідження:} лапароскопічні та відкриті резекції печінки


\textbf{Методи дослідження:} клініко–лабораторні, інструментальні методи (ультразвукове дослідження, мультидетекторна спіральна комп’ютерна томографія, магніторезонансна томографія, волюметрія печінки та черевної порожнини, зовнішнє вимірювання розмірів черевної порожнини), статистичні.


\section{Наукова новизна отриманих результатів.} 

Дисертаційна робота містить новий підхід до вирішення актуальної наукової проблеми щодо оптимізації хірургічного лікування пацієнтів з пухлинами печінки шляхом застосування лапароскопічного методу резекції. У даному дослідженні вперше буде встановлено патогенетичні особливості як лише злоякісних, так і доброякісних пухлин печінки, що впливають на вибір хірургічної тактики. Це дозволить покращити розуміння механізму росту пухлин, їх взаємодії з навколишніми тканинами та васкуляризації, що є ключовими факторами при проведенні малоінвазивних хірургічних втручань.
Буде проведено комплексний аналіз діагностичних можливостей сучасних методів обстеження пацієнтів із пухлинами печінки, таких як мультифазна комп’ютерна томографія, магнітно-резонансна томографія з контрастуванням та ультразвукове дослідження. На основі отриманих результатів буде розроблено стандартизований алгоритм передопераційного обстеження для визначення оптимального типу хірургічного втручання.
У рамках дослідження вперше буде детально оцінено вплив розміру пухлини, її анатомічного розташування, ступеня залучення судинних структур та стадії захворювання на можливість та доцільність застосування лапароскопічної резекції. Це сприятиме створенню уніфікованих критеріїв для відбору пацієнтів, яким лапароскопічне втручання буде найбільш ефективним та безпечним.
Також буде вдосконалено техніку виконання лапароскопічних резекцій печінки шляхом впровадження нових технічних прийомів, таких як селективна васкулярна ексклюзія, покращені методи хірургічної експозиції та транссекції печінки. Очікується, що ці інновації дозволять зменшити травматичність операцій, мінімізувати інтраопераційну крововтрату, скоротити тривалість втручання та підвищити точність маніпуляцій у складних анатомічних умовах.
Наукова новизна також включає проведення порівняльного аналізу результатів лапароскопічних та відкритих резекцій печінки. Буде досліджено частоту та характер післяопераційних ускладнень, тривалість госпіталізації, період відновлення функціональної активності пацієнтів, а також проведено аналіз якості їхнього життя до та після втручання. Оцінка якості життя здійснюватиметься за допомогою стандартизованих опитувальників, що дозволить об’єктивно порівняти функціональні результати різних методів хірургічного лікування.
У дослідженні буде вперше проведено диференційваний аналіз довгострокової онкологічної ефективності лапароскопічних резекцій печінки у порівнянні з відкритими операціями. Особливу увагу буде приділено вивченню частоти рецидивів, безрецидивного та загального виживання пацієнтів. Буде визначено ключові прогностичні фактори, що впливають на результати лікування, включаючи особливості пухлини, тип хірургічного доступу та клініко-патологічні характеристики пацієнтів.
На основі отриманих даних буде вперше розроблено алгоритм вибору хірургічної тактики для пацієнтів з пухлинами печінки. Він враховуватиме індивідуальні особливості захворювання, технічні можливості медичних закладів та переваги лапароскопічного підходу. Це дозволить підвищити ефективність лікування, знизити ризик ускладнень та покращити якість життя пацієнтів, що зробить значний внесок у розвиток сучасної хірургії печінки.


\section{Практичне значення отриманих результатів} 

Скорочення післяопераційного періоду та підвищення якості життя пацієнтів забезпечить зниження рівня післяопераційного болю, швидше відновлення функціональної активності, скорочення терміну госпіталізації та зниження частоти післяопераційних ускладнень, що дозволить рекомендувати цей підхід для ширшого застосування. Аналіз довгострокових результатів лікування, включаючи безрецидивну та загальну виживаність пацієнтів після лапароскопічних резекцій, дозволить розробити рекомендації щодо вибору оптимального хірургічного доступу з урахуванням онкологічних результатів.
Розроблений алгоритм вибору хірургічної тактики дозволить стандартизувати підхід до лікування пухлин печінки, враховуючи розмір, локалізацію, васкуляризацію пухлини, а також загальний стан пацієнта. Це забезпечить індивідуалізований підхід до кожного випадку та підвищить ефективність лікування. Отримані дані можуть бути використані в онкохірургічних та гепатобіліарних центрах для вдосконалення тактики лапароскопічних резекцій печінки, покращення післяопераційного менеджменту та розробки нових клінічних протоколів.
Результати дослідження також можуть бути використані для підготовки лікарів-хірургів, навчальних програм із лапароскопічної хірургії, а також у створенні рекомендацій для хірургічних товариств і клінічних протоколів. Таким чином, практичне значення отриманих результатів полягає у підвищенні ефективності та безпеки хірургічного лікування пухлин печінки, що сприятиме покращенню якості медичної допомоги та результатів лікування пацієнтів.


\section{Особистий внесок здобувача}
Дисертаційна робота є результатом самостійної наукової діяльності автора. Здобувач самостійно визначив мету та завдання дослідження, обрав оптимальні методи його проведення, здійснив патентний пошук і провів ґрунтовний аналіз літературних джерел за темою роботи.

Дисертант особисто зібрав клінічний матеріал та виконав його аналіз, застосовуючи сучасні методи статистичної обробки даних. Усі результати, представлені в дисертації, отримані автором самостійно. Він безпосередньо брав участь у більшості оперативних втручань.

Здобувач самостійно розробив та впровадив у клінічну практику нові методи хірургічного лікування, що підтверджено патентами України на винахід. Також ним була створена сучасна діагностично-лікувальна тактика для ведення пацієнтів із підозрою на нейроендокринні пухлини підшлункової залози.

Автор здійснив аналіз отриманих даних, провів їх теоретичне узагальнення, сформулював висновки та розробив практичні рекомендації для клінічного застосування.


\section{Апробація результатів дисертації} Основні положення дисертаційної роботи та результати дослідження викладені й обговорені на:

.


\section{Публікації за темою дисертації.} За матеріалами дисертації опубліковані \colorbox{yellow}{ХХХХ} наукові праці, в тому числі \colorbox{yellow}{ХХХХ}  – у вигляді статей у фахових журналах, рекомендованих МОН України, \colorbox{yellow}{ХХХХ}  – у вигляді тез і доповідей у матеріалах вітчизняних та міжнародних з’їздів, конгресів та конференцій. Видані методичні рекомендації для хірургів, інтернів та студентів старших курсів медичних університетів.

\section{Обсяг та структура дисертації.} Дисертаційна робота викладена на \pageref{LastPage} сторінках друкованого тексту, складається з вступу, огляду літератури, \totchap\hspace{1pt} розділів власних досліджень, аналізу й узагальнення отриманих результатів, висновків, списку використаних джерел літератури. Дисертація ілюстрована \tottab\hspace{1pt} таблицею, \totfig\hspace{1pt} рисунками. Список використаних джерел літератури містить \total{citenum}\hspace{1pt}
посилань, в тому числі  – кирилицею,   – латиною.