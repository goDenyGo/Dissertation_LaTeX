\chapter{ВСТУП}



\textit{Актуальність теми.} 




\textbf{Зв’язок роботи з науковими програмами, планами, темами.}
Дисертаційна робота виконана відповідно до тематичного плану науково- дослідних робіт Національного інституту хірургії та трансплантології імені О. О. Шалімова НАМН України і є фрагментом комплексних тем: «Розробити методи діагностики та лікування хворих з нейроендокринними пухлинами підшлункової залози» (номер державної реєстрації 0111U002779), «Вивчити та обґрунтувати хірургічну тактику лікування хворих з нейроендокринними пухлинами підшлункової залози» (номер державної реєстрації 0113U006512).

\textbf{Мета і завдання дослідження.} Метою дослідження є покращення результатів хірургічного лікування пацієнтів з вогнищевою патологією печінки шляхом оцінки результатів застосування лапароскопічної резекції печінки та розробки диференційованої тактики вибору хірургічного доступу. 

Відповідно до поставленої мети сформульовані наступні \textbf{завдання дослідження.}
\begin{enumerate}
    \item Дослідити механізм ефективності лапароскопічних резекцій печінки, вивчити їх вплив на періопераційні показники
    \item Оцінити технічну складність відкритих та лапароскопічних резекцій печінки в залежності від анатомічних умов, нозології та необхідного об’єму резекції
    \item Розробити універсальний спосіб вибору точок постановки троакарів під час проведення лапароскопічної резекції печінки
    \item Удосконалити методику проведення лапароскопічних резекцій печінки шляхом розробки уніфікованих показів до вибору оперативного доступу, способу мобілізації печінки, та транссекції паренхіми
    \item Дослідити в порівняльному аспекті частоту, можливі причини розвитку та специфіку ранніх та віддалених ускладнень після відкритої та лапароскопічної резекції печінки, розробити заходи щодо їх профілактики, вдосконалити методи діагностики та лікування
    \item Дослідити вплив лапароскопічних резекцій печінки на віддалені онкологічні  результати у пацієнтів із злоякісними новоутвореннями печінки
    \item Дослідити якість життя хворих на вогнищеву патологію печінки залежно від виду перенесеної рзезекції̈ печінки
    \item Оптимізувати тактику хірургічного лікування хворих на злоякісну та доброякісну патологію печінки шляхом розробки диференційованого підходу щодо вибору відкритого або лапароскопічного втручання в залежності від ступеню його складності та очікуваних переваг
\end{enumerate}


\textbf{Об’єкт дослідження:} Резектабельні форми доброякісної та злоякісної вогнищевої патології печінки

\textbf{Предмет дослідження:} Лапароскопічна резекція печінки


\textbf{Методи дослідження:} клініко–лабораторні, інструментальні методи (ультразвукове дослідження, мультидетекторна спіральна комп’ютерна томографія, магніторезонансна томографія, ендоскопічна ультрасонографія, сцинтиграфія рецепторів соматостатину, позитронно-емісійна комп’ютерна томографія, статистичні.


\textbf{Наукова новизна отриманих результатів.} 
Дослідженно вплив лапароскопічної резекції на післяопераціні результати хірургічного лікування пацієнтів із вогнищевою патологією печінки
Вивчено фактори що обумовлюють та на основі них запропоновано модель передопераційної  оцінки технічної складності лапароскопічної резекції печінки
Вивчено взаємозв’язок розташування троакарного доступу та складності виконання резекції. Обгрунтовано методику вибору точок постановки троакарів в залежності від планованого об’єму резекції
Вдосконалено метод етапної мобілізації лівої та правої долей печінки печінки.
На основі аналізу рентген-анатомічних характеристик вогнищевої патології печінки розроблено та обгрунтовано методику тракції під час проведення транссекції паренхіми
Вивчено фактори, що призводять до конверсій при виконанні лапароскопічних резекцій печінки та запропоновано способи їх профілактики 
Запропоновано алгоритм вибору між відкритим та лапароскопічним доступом при виконанні резекції печінки з приводу вогнищевої патології на основі оцінки прогнозованої технічної складності та післяопераційних ризиків 

\textbf{Практичне значення отриманих результатів.} Визначення інформативності специфічних лабораторних та інструментальних досліджень дозволило покращити якість передопераційної діагностики у пацієнтів з нейроендокринними пухлинами підшлункової залози. Впровадження розробленого сучасного діагностичного алгоритму у пацієнтів при припущенні на наявність нейроендокринних пухлин підшлункової залози та розробленої на його основі лікувальної тактики забезпечило покращення найближчих результатів хірургічного лікування пацієнтів, збільшення частоти виконання радикальних оперативних втручань з приводу нейроендокринних пухлин, покращення віддалених результатів.
Обґрунтування можливостей впровадження органозберігальних та нових мініінвазивних оперативних втручань у пацієнтів з приводу нейроендокринних пухлин підшлункової залози, дозволило зменшити частоту ендо- та екзокринної недостатності підшлункової залози в післяопераційному періоді та зменшити тривалість реабілітації пацієнтів після операції.
Впровадження системного підходу до попередження ранніх післяопераційних ускладнень у хворих з нейроендокринними пухлинами підшлункової залози, забезпечило достовірне зменшення частоти виникнення післяопераційних ускладнень.
Розроблення та впровадження нових методів оперативних втручань дозволило зменшити частоту ранніх післяопераційних ускладнень, об’єм інтраопераційної крововтрати, покращити віддалені результати лікування.
Вперше в Україні виконані лапароскопічна енуклеація інсуліноми підшлункової залози та інші види лапароскопічних втручань у пацієнтів з приводу нейроендокринних пухлин підшлункової залози.
Впровадження в клінічну практику оптимального діагностично- лікувального моніторингу пацієнтів з нейроендокринними пухлинами підшлункової залози, дозволило покращити доопераційну діагностику, вибрати оптимальний вид та метод оперативного втручання, покращити віддалені результати лікування.

\textbf{Особистий внесок здобувача.} Дисертаційна робота є самостійною працею автора. Здобувач самостійно визначив мету і завдання дослідження, обрав методи дослідження, провів патентний пошук та аналіз літератури за темою роботи. Дисертант самостійно зібрав клінічний матеріал та проаналізував його з використанням сучасних методів статистичної обробки. Викладені в роботі дані отримані автором особисто. Дисертант брав участь у більшості оперативних втручань.
Здобувач особисто розробив та впровадив в клінічну практику нові методи оперативних втручань, що підтверджено патентами України на винахід.
Дисертант особисто розробив сучасну діагностично-лікувальну тактику у пацієнтів з підозрою про наявність нейроендокринних пухлин підшлункової залози.
Автор проаналізував та теоретично узагальнив результати проведених досліджень, обґрунтував висновки та практичні рекомендації.


\textbf{Апробація результатів дисертації.} Основні положення дисертаційної роботи та результати дослідження викладені й обговорені на: 7 Annual ENETs Conference for the Diagnosis and Treatment of Neuroendocrine Tumor Disease (Berlin, Germany, 2010); ХХІІ з’їзді хірургів України (Вінниця, 2010);
8 Annual ENETs Conference for the Diagnosis and Treatment of Neuroendocrine Tumor Disease (Lisbon, Portugal, 2011); 10 Annual ENETs Conference for the Diagnosis and Treatment of Neuroendocrine Tumor Disease (Barselona, Spain, 2013); науково-практичній конференції з міжнародною участю «Актуальні проблеми клінічної хірургії» (Київ, 2013); XII Congress of the Association of Surgeons «Nicolae Anestiadi» from Republic of Moldova (Chisinau, Moldova, 2015); науково-практичній конференції з міжнародною участю «Актуальні питання невідкладної хірургії» (Харків, 2016); ХІІІ з’їзді онкологів та радіологів України (Київ, 2016).


\textbf{Публікації за темою дисертації.} За матеріалами дисертації опубліковані 42 наукові праці, в тому числі 23 – у вигляді статей у фахових журналах, рекомендованих МОН України, 14 – у вигляді тез і доповідей у матеріалах вітчизняних та міжнародних з’їздів, конгресів та конференцій. Отримані 4 патенти України на винахід. Видані методичні рекомендації для хірургів, інтернів та студентів старших курсів медичних університетів.

\textbf{Обсяг та структура дисертації.} Дисертаційна робота викладена на 281 сторінках друкованого тексту, складається з вступу, огляду літератури, 6 розділів власних досліджень, аналізу й узагальнення отриманих результатів, висновків, списку використаних джерел літератури. Дисертація ілюстрована 23 таблицею, 54 рисунками. Список використаних джерел літератури містить 338 посилань, в тому числі 12 – кирилицею, 326 – латиною.