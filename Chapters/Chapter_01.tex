\chapter {СУЧАСНІ ПРИНЦИПИ ЗАСТОСУВАННЯ ЛАПА\-РО\-СКО\-ПІ\-ЧНИХ РЕЗЕКЦІЙ В ХІРУРГІЧНОМУ ЛІКУВАННІ ВОГНИЩЕВОЇ ПАТОЛОГІЇ ПЕЧІНКИ (огляд літератури)}

\section{Історія розвитку методики}
\subsection{Перший досвід}
Мініінвазивні втручання радикально змінили хірургічну практику за останні три десятиріччя. Їх поява призвела до суттєвого покращення результатів за рахунок зменшення частоти післяопераційних ускладненнь, тривалості госпіталізації та співвідношення вартості до еффективності лікування.  Ці зміни торкнулись багатьох хірургічних спеціальностей, включаючи колоректальну хірургію, урологію, гінекологію та торокальну хірургію. Природньо, що зацікавленість хірургів в лапароскопічному доступі швидко розповсюдилась на гепатобіліарні втручання, спроби виконання яких в лапароскопічному варіанті було розпочато в 1987 році із першої лапароскопічної холецистектомії \cite{Litynski}. 

Для лікування уражень печінки лапароскопія була вперше впроваджена на початку 90х. На відміну від інших галузей, в печінковій хірургії лапароскопія зіткнулась із багатьма технічними перешкодами, пов'язаними із складною внутрішньою будовою та вразливою, схильною до кровотеч паренхімою печінки. Проте переваги лапароскопічного доступу, такі як покращена візуалізація, та зменшення післяопераційних ускладненнь, надали стимул для розвитку лапароскопічних резекцій печінки та здолання цих перешкод (\acrshort{llr}). 

Перші резекції печінки в лапароскопічному варіанті були виконані в 1991 році Reich H. \cite{Reich1991a} та  в 1992 Katkhouda N. \cite{Katkhouda1992} та Gagner M. \cite{GAGNER1992}. Ці операції були крайовими резекціями невеликих, переважно доброякісних, новоутворень. Невдовзі стало зрозуміло, що результати лапароскопічних операцій порівняні із традиційними відкритими втручаннями а \acrshort{llr} є безпечним та ефективним методом лікування. Відтоді \acrshort{llr} стала потенційною альтернативою відкритій резекції печінки (\acrshort{olr}). 

На початковому етапі розвиток методики був повільним, і наступні декілька років після першої згадки про \acrshort{llr} публікації містили лише опис поодиноких випадків атипових резекцій \cite{Klotz1993, Cunningham1995}. Перша анатомічна лапароскопічна лівобічна латеральна секцієектомія(\acrshort{llls}) виконана в 1996 році \cite{Azagra1996}  відновила інтерес до \acrshort{llr}  \cite{Hashizume1995}.  Із накопиченням досвіду покази до \acrshort{llr} були розширені до гемігепатектомій, складних сегментектомій, трисекцієектомій та навіть донорських резекцій при трансплантації печінки від живого донора \cite{Dagher2009, Cherqui2002, Jia2018}. 

\subsection{Погоджувальні конференції}

Від моменту виконання першої \acrshort{llr} кількість публікацій присвячених темі щорічно прогресивно збільшується (Рис. \ref{fig:PubMed_publications}). Стимулом до цього є посітйне вдосконалення ендохірургічного обладнання та хірургічної техніки.

Набутий досвід створив необхідність оцінки хірургічною спільнотою безпеки, відтворюваності та онкологічної ефективності \acrshort{llr}. Для вирішення цих завданнь та створення клінічних рекомендацій стосовно застосування \acrshort{llr} було  проведено декілька погоджувальних конференцій, висновки яких відображують процес становлення лапароскопічного методу та зміну відношення до нього хірургічного загалу. 

\begin{figure}[h]
\caption{Динаміка кількості публікацій в PubMed по запиту "laparosopic liver resection" по роках \cite{Hashizume1995}}
\centering
\includegraphics[width=0.9\textwidth]{Illustrations/PubMed_publications.jpg}
\label{fig:PubMed_publications}
\end{figure}

