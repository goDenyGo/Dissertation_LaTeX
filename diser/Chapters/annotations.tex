Головинський А.Л. Розвиток архітектури та базового програмного забезпечення високопродуктивних обчислювальних комплексів. -- Рукопис.

Дисертація на здобуття наукового ступеня кандидата технічних наук за спеціальністю 05.13.05 -- комп'ютерні системи та компоненти. -- Інститут кібернетики імені В.М.~Глушкова НАН України, Київ, 2012.

Дисертаційна робота присвячена питанням розвитку архітектури та базового програмного забезпечення високопродуктивних обчислювальних комплексів.

У роботі розв'язано задачі проектування архітектури обчислювального кластера, а саме завантаження всіх вузлів кластера з єдиного образу операційної системи, підвищення ефективності роботи керівних серверів кластера шляхом використання технологій віртуалізації. Це дозволило підвищити відмовостійкість, зменшити вартість побудови відповідних підсистем кластера.

Також розв'язано задачу проектування гібридного вузла для суперкомп'ютера нового покоління на основі потокових процесорів графічних адаптерів. Показано, що гібридний вузол є високопродуктивним обчислювальним елементом, придатним для побудови великих кластерних систем.

Розв'язано задачі розробки інтерфейсу керування кластером та роботи у гріді, методики збору та аналізу статистичних даних про використання ресурсів, моделі появи та виконання на обчислювальних задач та методів оцінки завантаження черги задач кластера.

Розроблено систему керування високопродуктивним обчислювальним комплексом, яка дозволила створити інтелектуальні засоби керування, моніторингу та діагностики структурних елементів комплексів, для яких розроблено програмні засоби для підвищення надійності та ефективності кластерів.

Запропоновано способи енергозбереження кластера шляхом динамічного регулювання кількості доступних користувачам обчислювальних ресурсів.

Архітектурні рішення, методи та програмне забезпечення, розроблені в результаті даного дисертаційного дослідження, були використані при побудові суперкомп'ютерного комплексу СКІТ Інституту кібернетики ім. В.М.~Глушкова НАН України, низки обчислювальних кластерів в Україні та за її межами.

\textit{Ключові слова:} базове програмне забезпечення суперкомп'ютера, архітектура суперкомп'ютера, кластерні системи, віртуалізація, енергозбереження, керування ресурсами, статистика використання суперкомп'ютера, безпека кластера.

\vspace{0.5em}
\begin{center}
 \textbf{АННОТАЦИЯ}
\end{center}
% \vspace{1em}

Головинский А.Л. Развитие архитектуры и базового программного обеспечения высокопроизводительных вычислительных комплексов. -- Рукопись.

Диссертация на соискание научной степени кандидата технических наук по специальности 05.13.05 -- компьютерные системы и компоненты. -- Институт кибернетики имени В.М.~Глушкова НАН Украины, Киев, 2012.

Диссертация посвящена вопросам развития архитектуры и базового программного обеспечения высокопроизводительных вычислительных комплексов.

В работе решены задачи проектирования архитектуры вычислительного кластера. А именно, разработана система загрузки всех узлов кластера с единого образа операционной системы. Это позволило избежать проблем синхронизации образов, снизить стоимость построения и повысить надежность за счет упрощения дисковой подсистемы узлов.

Решена задача повышения эффективности работы управляющих серверов кластера путем использования технологий виртуализации. Это позволило перенести большинство серверов в виртуальные контейнеры и, таким образом, повысить отказоустойчивость соответствующих сервисов, в 4--5 раз уменьшить стоимость построения и энергопотребления управляющей подсистемы кластера.

Также решена задача проектирования гибридного узла для суперкомпьютера нового поколения на основе потоковых процессоров графических адаптеров. Показано, что гибридный узел является высокопроизводительным вычислительным элементом, пригодным для построения больших кластерных систем.

Решены задачи разработки интерфейса управления кластером и работы в гриде, методики сбора и анализа статистических данных об использовании ресурсов, модели появления и исполнения вычислительных задач и методов оценки загрузки очереди задач кластера.

Разработана система управления высокопроизводительным вычислительным комплексом, которая позволила создать интеллектуальные средства управления, мониторинга и диагностики его структурных элементов.

В работе предложены способы энергосбережения кластера путем динамического регулирования количества доступных пользователям вычислительных ресурсов.

Для обеспечения круглосуточного безотказного функционирования кластера разработаны методы самодиагностики, активные системы самовосстановления и защиты.

Архитектурные решения, методы и программы, разработанные в результате данного исследования, были использованы при построении суперкомпьютерного комплекса СКИТ Института кибернетики имени В.М.~Глушкова НАН Украины, ряда вычислительных кластеров в Украине и за ее пределами.

\textit{Ключевые слова:} базовое программное обеспечение суперкомпьютера, архитектура суперкомпьютера, кластерные системы, виртуализация, энергосбережение, управление ресурсами, статистика использования суперкомпьютера, безопасность кластера.

\vspace{0.5em}
\begin{center}
\textbf{ABSTRACT}
\end{center}
% \vspace{1em}

Golovynskyi A.L. Development of architecture and basic software for high performance computing systems. -- Manuscript.

Thesis for Candidate degree in speciality 05.13.05 -- computer systems and com\-po\-ne\-nts. -- Glushkov Institute of Cybernetics NAS of Ukraine, Kyiv, 2012.

Thesis is devoted to the development of architecture and basic software for high performance computing systems. 

The paper proposed cluster architecture improvements, such as downloading all the cluster nodes from a single operating system image, improving effectivness of management servers by using virtualization technologies

Also solved the problem of designing a hybrid nodes for a new generation of supercomputers with stream processors based on graphics adapters. It is shown that such hybrid nodes are suitable for building large cluster systems.

The paper proposed methods for building energy-efficient supercomputer through dynamic regulation of the number of available computing resources for users, user activity and task flow prediction model, methods for selecting the best task scheduling algorithm for the specific cluster.

Round the clock failsafe supercomputer functioning, ar\-chi\-tec\-tu\-ral solutions, methods for self-diagnosis, active self-healing and protection systems were designed.

Architectural solutions, methods and software developed as a result of this disser\-tation research were used during construction of the supercomputer SKIT in Glushkov Institute of Cybernetics and number of computing clusters in Ukraine and abroad.


\textit{Key words:} basic supercomputer software, supercomputer architecture, cluster systems, virtualization, energy efficiency, resource management, supercomputer usage statistics, cluster security. 


